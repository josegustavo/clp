%%%%%%%%%%%%%%%%%%%%%%%%%%%%%%%%%%%%%%%%%%%%%%%%%%%%%%%%%%%
% --------------------------------------------------------
% Rho
% LaTeX Template
% Version 1.0 (28/04/2024)
%
% Author: 
% Guillermo Jimenez (memo.notess1@gmail.com)
% 
% License:
% Creative Commons CC BY 4.0
% --------------------------------------------------------
%%%%%%%%%%%%%%%%%%%%%%%%%%%%%%%%%%%%%%%%%%%%%%%%%%%%%%%%%%%
% --------------------------------------------------------
%					  FOR SPANISH BABEL
% --------------------------------------------------------
%\usepackage[spanish,es-nodecimaldot,es-noindentfirst]{babel}
% --------------------------------------------------------
%%%%%%%%%%%%%%%%%%%%%%%%%%%%%%%%%%%%%%%%%%%%%%%%%%%%%%%%%%%

\documentclass[9pt,a4paper]{rho}
\usepackage[utf8]{inputenc}
\usepackage[T1]{fontenc}    % Codificación de fuentes
\usepackage[spanish]{babel} % Configuración del idioma
\usepackage{csquotes}       % Recomendado para idiomas que usan comillas
\usepackage{rhoenvs}

%----------------------------------------------------------
% TITLE
%----------------------------------------------------------

\title{Optimización del Llenado de Contenedores con un Sistema de Soporte de Decisiones}

%----------------------------------------------------------
% AUTHORS AND AFFILIATIONS
%----------------------------------------------------------

\author[$\dagger$]{Laura J. Rodriguez}
\author[$\dagger$]{Jose G. Quilca}

%----------------------------------------------------------

\affil[$\dagger$]{Centro de Investigación Operativa, Universidad Miguel Hernández, 03202 Elche, Alicante, España}

%----------------------------------------------------------
% DATES
%----------------------------------------------------------

\dates{\textbf{Asignatura}: Sistemas de Apoyo a la Decisión\\
\textbf{Práctica 2}: Diseño Funcional de un DSS}

%----------------------------------------------------------
% FOOTER INFORMATION
%----------------------------------------------------------

\etal{Máster Universitario en Estadística Computacional y Ciencia de Datos para la Toma de Decisiones}
\footinfo{Curso 2023-2024}
\smalltitle{UMH}
\institution{Sistemas de Apoyo a la Decisión}
\date{21 de Mayo de 2024} %\today

%----------------------------------------------------------
% ARTICLE INFORMATION
%----------------------------------------------------------

%----------------------------------------------------------
% ABSTRACT
%----------------------------------------------------------

\begin{abstract}
    El presente documento detalla el desarrollo de un Sistema de Soporte de Decisiones (DSS) destinado a optimizar el llenado de contenedores marítimos con paquetes de diferentes dimensiones, pesos y valores. Este sistema, basado en algoritmos genéticos y técnicas de simulación, busca maximizar el uso del espacio y el valor total de la carga, priorizando según la importancia económica de los productos. La implementación de este DSS permite a las empresas mejorar la eficiencia logística, reducir costos operacionales y aumentar la satisfacción del cliente mediante una entrega más rápida y confiable. El prototipo desarrollado demuestra la viabilidad del sistema, mostrando mejoras significativas en el llenado de contenedores y estableciendo una base sólida para futuros desarrollos y despliegues en entornos operativos reales.
\end{abstract}

%----------------------------------------------------------

\keywords{Sistema de Soporte de Decisiones, optimización de contenedores, algoritmos genéticos, logística, simulación de llenado, eficiencia operativa}

%----------------------------------------------------------

\begin{document}

\maketitle
\thispagestyle{firststyle}
% \tableofcontents

%----------------------------------------------------------

\section{Entorno de aplicación del DSS}

Para la dinamización del comercio internacional, las estrategias logísticas y de transporte juegan un papel crucial en el éxito operacional y financiero de las empresas que distribuyen sus productos a nivel global. Según \textcite{rushton2017international}, un aspecto fundamental en este ámbito es la eficiencia en el llenado y transporte de contenedores. La optimización de espacios dentro de estos contenedores puede significar ahorros significativos o pérdidas considerables en términos de costos y satisfacción del servicio al cliente, como indican \textcite{bortfeldt2013container}.

El presente documento detalla el desarrollo de un Sistema de Apoyo a la Decisión \cite{jao2010decision}, cuyo objetivo es proporcionar soporte a una empresa que tiene entre sus operaciones críticas el envío de productos en contenedores buscando la mejor manera de completar contenedores con cajas y paquetes, con el fin de optimizar tanto la carga como el valor de los productos enviados. Este sistema no solo busca maximizar el uso del espacio dentro de los contenedores, sino también priorizar la carga según la importancia económica de los productos. Al utilizar el DSS elaborado, se pretende proporcionar soluciones basadas en datos que mejoren las decisiones de carga y envío, reduciendo los costos operacionales y aumentando la eficiencia general del proceso logístico \cite{simchi2003logistics}.

Este trabajo presenta una estructura sólida que explica detalladamente el entorno de aplicación del DSS, los componentes involucrados en el sistema y el proceso de desarrollo e implementación llevado a cabo para construir una herramienta avanzada de apoyo a la decisión. Según \textcite{christopher2016logistics}, una empresa será capaz de enfrentar los desafíos del mercado actual, donde la precisión y rapidez de la información en el ámbito logístico son de suma importancia.

El potencial significativo de la implementación de este sistema transformará completamente la manera en que una empresa opera en cuanto a la carga de paquetes y envíos programados, convirtiéndola en un referente en términos de innovación y eficiencia en este sector. La adopción de este DSS permitirá a una empresa optimizar sus operaciones logísticas, mejorar la satisfacción del cliente mediante una entrega más rápida y confiable, y mantener una ventaja competitiva en el mercado global, como indican \parencite{slack2010operations}.

\section{Optimización de la Carga de Contenedores}

Para desarrollar un Sistema de Soporte de Decisiones (DSS) que calcule el llenado óptimo de un contenedor con paquetes, es crucial definir claramente los problemas decisionales que este sistema debe resolver, según \textcite{murty2005decision}. Estos problemas decisionales son aquellos desafíos y decisiones que el sistema debe abordar para proporcionar soluciones óptimas. A continuación se detallan los principales problemas decisionales a resolver:

\subsection{Selección de Paquetes para Maximizar el Espacio Utilizado}

\textbf{Problema:} Dado un conjunto de paquetes de diferentes dimensiones y pesos, seleccionar cuáles incluir en el contenedor para maximizar el uso del espacio disponible. Según \textcite{gilmore1965multistage}, el problema consiste en seleccionar cuáles paquetes incluir en el contenedor para maximizar el uso del espacio disponible, dado un conjunto de paquetes de diferentes dimensiones y pesos.

\textbf{Consideraciones:}
\begin{itemize}
    \item Dimensiones del contenedor (largo, ancho, alto).
    \item Dimensiones y peso de cada paquete.
    \item Restricciones de peso y volumen del contenedor.
    \item Posibilidad de apilar paquetes y restricciones relacionadas.
\end{itemize}

\subsection{Optimización de la Configuración de Paquetes dentro del Contenedor}

\textbf{Problema:} Determinar la disposición óptima de los paquetes seleccionados dentro del contenedor para maximizar el uso del espacio. El problema de determinar la disposición óptima de los paquetes seleccionados dentro del contenedor para maximizar el uso del espacio ha sido estudiado por \textcite{zhao2016comparative}.

\textbf{Consideraciones:}
\begin{itemize}
    \item Orientación de los paquetes (rotación en 3D).
    \item Estabilidad de la carga (evitar que paquetes más pesados estén sobre más ligeros).
    \item Accesibilidad de los paquetes (si se requiere acceso a ciertos paquetes antes que a otros).
\end{itemize}

\subsection{Minimización del Tiempo de Carga y Descarga}

\textbf{Problema:} Optimizar la disposición de los paquetes para reducir el tiempo necesario para cargar y descargar el contenedor. El problema de optimizar la disposición de los paquetes para reducir el tiempo necesario para cargar y descargar el contenedor es abordado por \textcite{bastidas}.

\textbf{Consideraciones:}
\begin{itemize}
    \item Orden de carga y descarga según el destino de los paquetes.
    \item Acceso rápido a los paquetes que deben ser descargados primero.
    \item Facilitar la manipulación de los paquetes durante la carga y descarga.
\end{itemize}

Estos problemas decisionales deben ser abordados por el DSS para ofrecer una solución integral que no solo maximice la eficiencia del llenado del contenedor, sino que también cumpla con todas las restricciones y necesidades operativas. Al implementar algoritmos de optimización y técnicas de simulación, el DSS puede proporcionar recomendaciones y planes de acción basados en los datos disponibles, ayudando así a mejorar la eficiencia logística y la toma de decisiones en el proceso de carga y transporte de mercancías.

\section{Componentes del DSS Propuesto}

\subsection{Esquema General del DSS Propuesto}

El sistema de soporte de decisiones (DSS) propuesto está diseñado para optimizar el llenado de contenedores marítimos de diversos tamaños con paquetes de diferentes dimensiones, pesos y valores. El objetivo principal del DSS es maximizar el valor total de la carga y optimizar el uso del volumen del contenedor, asegurando que todos los paquetes del mismo tipo se agrupen para facilitar la carga y descarga. El sistema utiliza un algoritmo genético \cite{bortfeldt2001hybrid} para encontrar la mejor disposición de los paquetes dentro del contenedor.

El flujo general del sistema comienza con la entrada de datos, donde el usuario ingresa las dimensiones y el peso máximo del contenedor, junto con las dimensiones, pesos, valores y cantidades disponibles de los paquetes. A continuación, el usuario selecciona los parámetros del algoritmo genético. Durante la fase de optimización, el algoritmo genético, en conjunto con un motor de simulación de llenado manual, busca la mejor solución dentro de un tiempo definido. Al finalizar la optimización, el sistema genera una disposición optimizada de los paquetes en el contenedor y un manual de llenado que indica el orden de llenado de cada tipo de caja. Finalmente, el usuario utiliza el reporte final para realizar el llenado manual del contenedor.

A continuación se presenta un diagrama de componentes que ilustra la interacción entre los diferentes elementos del DSS propuesto:

\begin{figure}[h!]
    \centering
    \includegraphics[width=0.46\textwidth]{component_diagram.png}
    \caption{Diagrama de componentes del Sistema de Soporte de Decisiones (DSS).}
    \label{fig:component_diagram}
\end{figure}

El diagrama de componentes del Sistema de Soporte de Decisiones (DSS) muestra cómo interactúan entre sí los diferentes elementos del sistema para optimizar el llenado de contenedores marítimos.

\begin{itemize}
    \item \textbf{Usuario:} El usuario, representado como "Planificador de Carga", interactúa con el sistema a través de la Interfaz de Usuario (Shiny para Python \cite{shiny_intro}). El usuario ingresa las dimensiones y el peso máximo del contenedor, así como las dimensiones, pesos, valores y cantidades disponibles de los paquetes. También configura los parámetros del algoritmo de optimización.
    \item \textbf{Interfaz de Usuario (Shiny):} Este componente permite al usuario interactuar con el sistema. La interfaz se utiliza para ingresar datos, configurar el algoritmo, monitorear la optimización y recibir los resultados finales. La interfaz envía los datos ingresados al Algoritmo Genético y recibe los resultados finales de optimización.
    \item \textbf{Algoritmo Genético:} Este componente es el motor principal de optimización. Toma los datos de entrada proporcionados por el usuario y busca la mejor solución de llenado del contenedor dentro de un tiempo definido. El algoritmo genético interactúa con el Motor de Simulación de Llenado Manual para validar y evaluar las soluciones generadas. Además, el algoritmo genético accede a la Base de Datos para leer y escribir datos relevantes.
    \item \textbf{Motor de Simulación de Llenado Manual:} Este motor simula el proceso de llenado manual del contenedor, considerando las limitaciones prácticas y operativas. Proporciona feedback continuo al algoritmo genético para mejorar la optimización en cada iteración. También interactúa con la Base de Datos para leer configuraciones necesarias.
    \item \textbf{Base de Datos:} Este componente almacena y gestiona los datos del sistema, incluyendo configuraciones de contenedores, tipos de paquetes y resultados de optimización previos. La base de datos permite la lectura y escritura de datos por parte del algoritmo genético y el motor de simulación de llenado manual.
\end{itemize}

El flujo general del sistema comienza con la entrada de datos por parte del usuario. Estos datos se envían al algoritmo genético, que en conjunto con el motor de simulación busca la mejor solución de llenado. La base de datos almacena y gestiona todos los datos necesarios. Finalmente, el usuario recibe los resultados a través de la interfaz de usuario, incluyendo una disposición optimizada de los paquetes y un manual de llenado detallado.

\subsection{Bases de Datos de Entrada}

El sistema puede utilizar una base de datos \cite{elmasri2021fundamentals} Relacional para almacenar y gestionar los datos necesarios. Las bases de datos de entrada incluyen configuraciones de contenedores, que abarcan las dimensiones y pesos máximos de los contenedores marítimos utilizados en la industria, y tipos de paquetes, que comprenden las dimensiones, pesos, valores y cantidades disponibles de los paquetes, conectados al área de almacén para conocer la disponibilidad máxima. Además, se almacenan los resultados de optimización previos para identificar parámetros óptimos del algoritmo genético según distintas configuraciones. Todos los datos son estructurados, permitiendo una gestión eficiente y una alta tolerancia a fallos.

\subsection{Motores a Incorporar}

El DSS propuesto incorpora dos motores principales. El primero es el algoritmo genético, que toma los datos de entrada, busca la mejor solución dentro de un tiempo definido y devuelve el mejor resultado encontrado. Este motor es responsable de generar diversas combinaciones y evaluar su eficiencia. El segundo es el motor de simulación de llenado manual, que simula el proceso de llenado manual del contenedor, considerando limitaciones prácticas y operativas. Este motor evalúa la validez y el valor de las soluciones propuestas por el algoritmo genético. La interacción entre estos motores es fundamental: el algoritmo genético es el motor principal y utiliza el motor de simulación para validar y valorar las soluciones generadas. El motor de simulación proporciona feedback continuo al algoritmo genético para mejorar la optimización en cada iteración.

\subsection{Diseño Gráfico del Front-End}

El front-end del DSS está diseñado para ser intuitivo y eficiente, desarrollado con Shiny para Python y utilizando matplotlib para la generación de gráficos. La interfaz incluye varias funcionalidades clave. Los usuarios pueden seleccionar las configuraciones del contenedor, incluyendo las dimensiones y el peso máximo, y configurar los parámetros del algoritmo de llenado. Además, pueden ingresar las dimensiones, pesos, valores y cantidades disponibles de los paquetes. Durante el proceso de optimización, la interfaz permite monitorear la evolución de cada iteración del algoritmo genético en tiempo real. Una vez completada la optimización, el sistema ofrece una visualización en 3D del llenado simulado del contenedor. Finalmente, se genera un reporte detallado que incluye un manual de llenado, indicando el orden y las cantidades de llenado por tipo de paquete.

Para ofrecer una visión más clara del diseño del sistema, se presenta a continuación el wireframe del DSS propuesto:

\begin{figure}[h!]
    \centering
    \includegraphics[width=0.46\textwidth]{wireframe_dss.png}
    \caption{Wireframe del Sistema de Soporte de Decisiones (DSS).}
    \label{fig:wireframe_dss}
\end{figure}

El wireframe ilustra la disposición y funcionalidad de la interfaz de usuario del DSS, mostrando cómo los usuarios pueden ingresar datos, secciones para configurar el algoritmo, monitorear la optimización y visualizar los resultados finales.

Los usuarios finales del sistema son los planificadores de carga, quienes utilizarán la interfaz para ingresar datos y recibir los reportes finales, los cuales serán entregados a los operarios responsables del llenado manual del contenedor.

\section{Aspectos de Usabilidad y Casos de Uso}

\subsection{Aspectos de Usabilidad}
\textbf{Interfaz Intuitiva y Fácil de Usar:} La interfaz del DSS está diseñada con Shiny para Python, proporcionando una experiencia de usuario intuitiva. Los usuarios pueden ingresar datos, configurar parámetros y visualizar resultados sin necesidad de conocimientos técnicos avanzados. Los formularios y paneles de control están organizados de manera lógica y accesible \cite{shiny_intro}.

\textbf{Entradas de Datos Estructuradas:} La entrada de datos se realiza mediante formularios estructurados donde los usuarios pueden especificar las dimensiones, peso máximo y otros parámetros del contenedor, así como las características de los paquetes (dimensiones, peso, valor y cantidades disponibles).

\textbf{Visualización en Tiempo Real:} Durante la optimización, los usuarios pueden observar la evolución del algoritmo genético en tiempo real a través de gráficos generados con matplotlib \cite{matplotlib}. Esto permite a los usuarios monitorear el progreso y realizar ajustes si es necesario.

\textbf{Accesibilidad Multiplataforma:} La aplicación web desarrollada con Shiny para Python es accesible desde cualquier dispositivo con navegador web, incluyendo computadoras de escritorio, portátiles, tabletas y teléfonos inteligentes. Esto garantiza que los usuarios puedan acceder al sistema desde cualquier lugar.

\textbf{Opciones de Reporting Automático:} El sistema genera reportes automáticos detallados al finalizar la optimización. Estos reportes incluyen la disposición optimizada de los paquetes, gráficos de visualización en 3D y un manual de llenado que indica el orden de llenado de cada tipo de caja. Los reportes se pueden exportar en formatos PDF y Excel para facilitar su distribución y análisis posterior.

\subsection{Casos de Uso}
A continuación se presenta un diagrama de casos de uso que ilustra las interacciones entre los diferentes actores y las funcionalidades del DSS propuesto:

\begin{figure}[h!]
    \centering
    \includegraphics[width=0.47\textwidth]{use_case_diagram.png}
    \caption{Diagrama de casos de uso del Sistema de Soporte de Decisiones (DSS).}
    \label{fig:use_case_diagram}
\end{figure}

El diagrama de casos de uso muestra las interacciones entre los diferentes actores y las funcionalidades del sistema de soporte de decisiones (DSS) propuesto.

\textbf{Planificador de Carga:} Este usuario interactúa principalmente con el sistema para realizar las siguientes tareas:
\begin{itemize}
    \item \textbf{Ingreso de Datos y Configuración Inicial:} El planificador de carga ingresa las dimensiones del contenedor, el peso máximo permitido, y las características de los paquetes (dimensiones, peso, valor y cantidades disponibles). También configura los parámetros del algoritmo genético.
    \item \textbf{Ejecución de la Optimización:} Después de configurar los parámetros, el planificador de carga inicia el proceso de optimización, donde el algoritmo genético busca la mejor disposición de los paquetes dentro del contenedor.
    \item \textbf{Monitoreo en Tiempo Real:} Durante la optimización, el planificador de carga monitorea el progreso del algoritmo a través de gráficos en tiempo real que muestran la evolución de cada iteración y las mejoras en la optimización.
    \item \textbf{Generación de Reportes:} Al finalizar la optimización, el sistema genera un reporte detallado que incluye la disposición optimizada de los paquetes en el contenedor, visualizaciones en 3D del llenado simulado y un manual de llenado con instrucciones específicas.
\end{itemize}

\textbf{Operario:} Este usuario utiliza los reportes generados por el sistema para realizar el llenado manual del contenedor:
\begin{itemize}
    \item \textbf{Acceso y Uso de Reportes:} Los operarios reciben los reportes generados y utilizan las instrucciones claras sobre el orden de llenado de cada tipo de caja para maximizar la eficiencia del proceso de llenado manual del contenedor.
\end{itemize}

\textbf{Posibles Formatos y Utilidades de Entrada de Datos:}
\begin{itemize}
    \item \textbf{Formularios Web:} Los datos se ingresan a través de formularios web estructurados que permiten la entrada de dimensiones, peso, valor y cantidades disponibles de los paquetes, así como las características del contenedor.
    \item \textbf{Importación de Archivos:} Se permite la importación de datos desde archivos CSV o Excel, facilitando la integración con otros sistemas de gestión y evitando la entrada manual de grandes volúmenes de datos.
    \item \textbf{Conexión a Bases de Datos:} Integración con sistemas de gestión de inventarios para obtener datos actualizados sobre la disponibilidad y características de los paquetes.
\end{itemize}

\textbf{Salidas Gráficas Disponibles:}
\begin{itemize}
    \item \textbf{Gráficos en Tiempo Real:} Visualización del progreso del algoritmo genético en tiempo real, mostrando la mejora continua en la optimización del llenado. Por ejemplo, en la Figura \ref{fig:ag_tiempo_real} se muestra un gráfico de evolución de la función objetivo a lo largo de las generaciones del algoritmo genético en este caso usando distintos métodos de mejora.
    \item \textbf{Visualización en 3D:} Representación en 3D de la disposición optimizada de los paquetes dentro del contenedor, ayudando a los usuarios a visualizar cómo se vería el contenedor una vez lleno. Por ejemplo, en la Figura \ref{fig:3d_visualization} se muestra una visualización en 3D del llenado del contenedor.
    \item \textbf{Reportes Detallados:} Generación de reportes en PDF y Excel con gráficos y visualizaciones claras, incluyendo un manual de llenado con instrucciones paso a paso.
\end{itemize}

\begin{figure}[h!]
    \centering
    \includegraphics[width=0.47\textwidth]{ag_tiempo_real.png}
    \caption{Gráfico de evolución de la función objetivo a lo largo de las generaciones del algoritmo genético.}
    \label{fig:ag_tiempo_real}
\end{figure}

\begin{figure}[h!]
    \centering
    \includegraphics[width=0.47\textwidth]{3d_visualization.png}
    \caption{Visualización en 3D del llenado del contenedor.}
    \label{fig:3d_visualization}
\end{figure}

\textbf{Facilidades de Accesibilidad:}
\begin{itemize}
    \item \textbf{Compatibilidad Multiplataforma:} Acceso al sistema desde cualquier dispositivo con navegador web, incluyendo computadoras de escritorio, portátiles, tabletas y teléfonos inteligentes.
\end{itemize}

Con estos aspectos de usabilidad, formatos de entrada de datos, salidas gráficas y facilidades de accesibilidad, el DSS propuesto asegura una experiencia de usuario óptima y accesible, facilitando la optimización del llenado de contenedores y mejorando la eficiencia operativa.

\section{Implementación de Prototipo}

\subsection{Tecnologías y Herramientas}
El prototipo se desarrollará utilizando Python \cite{python-wikipedia} como lenguaje de programación principal. Python es conocido por su simplicidad y versatilidad, siendo adecuado para una amplia gama de aplicaciones, incluidas aquellas en ciencia de datos y desarrollo web. Para la interfaz de usuario se empleará Shiny para Python, mientras que Matplotlib \cite{matplotlib-intro} se utilizará para la generación de gráficos, incluyendo la visualización estática en 3D del llenado del contenedor. Los motores de optimización y simulación, así como la lógica general del sistema, se implementarán también en Python.

\subsection{Alcance del Prototipo}
El prototipo permitirá seleccionar las configuraciones del contenedor ingresando manualmente las dimensiones y el peso máximo. Los usuarios podrán configurar el tiempo de ejecución del algoritmo genético, pero otros parámetros no estarán disponibles en esta versión. Los tamaños de las cajas se ingresarán manualmente y se utilizarán valores aleatorios para facilitar el llenado. La visualización del resultado del llenado del contenedor se mostrará como un gráfico estático generado con Matplotlib, sin opción de rotación en tiempo real . El progreso del algoritmo genético se visualizará solo al finalizar la optimización, mostrando el progreso por cada generación. Se generará un reporte en formato de tabla que detallará las cantidades de paquetes por tipo y el orden de llenado, sin opciones de exportación o impresión. Estas limitaciones reflejan la naturaleza del prototipo funcional y su alcance reducido para fines demostrativos.

\subsection{Desarrollo del Prototipo}
El desarrollo del prototipo seguirá un ciclo iterativo de desarrollo, prueba y corrección. Aunque no se seguirá una metodología específica como Agile, el proceso será similar en términos de iteración y retroalimentación continua. El desarrollo se realizará utilizando Visual Studio Code como entorno de desarrollo integrado (IDE) y las pruebas iniciales se llevarán a cabo en un entorno local \cite{vscode-intro}. Para el despliegue del prototipo, se considerará el uso de herramientas que soporten Shiny para Python, aunque no se ha determinado una plataforma específica para este propósito.

En la figura \ref{fig:prototipo_diagrama} se muestra una visión general de la interfaz de usuario del prototipo.

\begin{figure}[h!]
    \centering
    \includegraphics[width=0.47\textwidth]{prototipo_diagrama.png}
    \caption{Diagrama general del prototipo del Sistema de Soporte de Decisiones (DSS).}
    \label{fig:prototipo_diagrama}
\end{figure}

A continuación se detalla en las siguientes imágenes la interfaz de usuario del prototipo, incluyendo la selección del contenedor, la configuración de los paquetes, la visualización en tiempo real del progreso del algoritmo genético, la visualización estática en 3D del llenado del contenedor, la tabla de resultados generada al finalizar la optimización, el resumen del estado actual de la optimización.

\begin{figure}[h!]
    \centering
    \includegraphics[width=0.2\textwidth]{prototipo_interfaz_1.png}
    \caption{Opciones de selección y configuración del contenedor en el prototipo.}
    \label{fig:prototipo_interfaz_1}
\end{figure}

En la figura \ref{fig:prototipo_interfaz_1} se muestra la interfaz de usuario del prototipo, donde se pueden seleccionar las configuraciones del contenedor ingresando manualmente las dimensiones como el largo, ancho y alto, y el peso máximo.

\begin{figure}[h!]
    \centering
    \includegraphics[width=0.47\textwidth]{prototipo_interfaz_2.png}
    \caption{Configuración de los tipos paquetes y sus cantidades.}
    \label{fig:prototipo_interfaz_2}
\end{figure}

En la figura \ref{fig:prototipo_interfaz_2} se muestra la configuración de los tipos de paquetes, donde se pueden ingresar manualmente las dimensiones (largo, ancho, alto), peso, valor de cada tipo de paquete y las cantidades máximas disponibles.

\begin{figure}[h!]
    \centering
    \includegraphics[width=0.47\textwidth]{prototipo_interfaz_3.png}
    \caption{Visualización en del progreso del algoritmo genético.}
    \label{fig:prototipo_interfaz_3}
\end{figure}

En la figura \ref{fig:prototipo_interfaz_3} se muestra la visualización del progreso del algoritmo genético, donde se puede observar la evolución de la función objetivo a lo largo de las generaciones.

\begin{figure}[h!]
    \centering
    \includegraphics[width=0.47\textwidth]{prototipo_interfaz_4.png}
    \caption{Visualización en 3D del llenado del contenedor.}
    \label{fig:prototipo_interfaz_4}
\end{figure}

En la figura \ref{fig:prototipo_interfaz_4} se muestra la visualización en 3D del llenado del contenedor, donde se puede observar la disposición optimizada de los paquetes dentro del contenedor. Por razones de simplicidad, la visualización es estática y no permite rotación en tiempo real.

\begin{figure}[h!]
    \centering
    \includegraphics[width=0.27\textwidth]{prototipo_interfaz_5.png}
    \caption{Tabla de resultados generada al finalizar la optimización.}
    \label{fig:prototipo_interfaz_5}
\end{figure}

En la figura \ref{fig:prototipo_interfaz_5} se muestra la tabla de resultados generada al finalizar la optimización, que detalla el orden de llenado, las cantidades de paquetes por tipo y si el paquete debe ser rotado o no.

\begin{figure}[h!]
    \centering
    \includegraphics[width=0.47\textwidth]{prototipo_interfaz_6.png}
    \caption{Tarjetas de resumen del estado actual de la optimización.}
    \label{fig:prototipo_interfaz_6}
\end{figure}

En la figura \ref{fig:prototipo_interfaz_6} se muestran las tarjetas de resumen del estado actual de la optimización, que proporcionan información sobre el valor total de la carga actual, el valor que se ha incrementado luego de la optimización y el porcentaje de llenado del contenedor.

\subsection{Criterios de Éxito}
El prototipo se considerará exitoso si logra demostrar la aplicación del DSS al problema de optimización del llenado de contenedores. Los criterios específicos de éxito \cite{newman2000success} incluyen la demostración clara de cómo el DSS ayuda a los usuarios a tomar mejores decisiones y la visualización de una mejora en el porcentaje de llenado comparado con una disposición inicial no optimizada.

\subsection{Pruebas y Validación}
Las pruebas del prototipo se realizarán de manera manual. Primero, se ingresarán manualmente las dimensiones del contenedor y los tamaños, pesos y valores de los paquetes. Se observará el valor y el porcentaje de llenado del contenedor antes de la optimización y luego se ejecutará el algoritmo genético para optimizar el llenado. Se evaluará la mejora en el porcentaje de llenado y se visualizará la simulación del llenado en un gráfico estático en 3D. Finalmente, se generará y revisará la tabla que detalla las cantidades de paquetes por tipo y el orden de llenado. Estas pruebas permitirán validar el funcionamiento del prototipo y su capacidad para mejorar el llenado del contenedor.

\section{Posibles Barreras de Entrada para su Implantación Real}

A pesar del potencial significativo del Sistema de Soporte de Decisiones (DSS) para la optimización del llenado de contenedores, existen varias barreras de entrada que podrían dificultar su implantación real en entornos operativos. A continuación se detallan algunas de las principales barreras:

\subsection{Costo de Implementación}
La implementación de un DSS de esta naturaleza puede requerir una inversión considerable en términos de desarrollo, integración con sistemas existentes y formación de personal. Las empresas deberán evaluar si los beneficios esperados justifican estos costos iniciales.

\subsection{Integración con Sistemas Existentes}
La integración del DSS con los sistemas de gestión de inventarios y logística ya en uso en la empresa puede ser compleja. Las empresas deben asegurarse de que los datos fluyan sin problemas entre los diferentes sistemas para que el DSS funcione de manera eficiente.

\subsection{Resistencia al Cambio}
La introducción de un nuevo sistema de soporte de decisiones puede enfrentar resistencia por parte del personal que está acostumbrado a los procesos actuales. Es crucial llevar a cabo una gestión del cambio efectiva, incluyendo formación y comunicación claras sobre los beneficios del nuevo sistema.

\subsection{Precisión y Calidad de los Datos}
El rendimiento del DSS depende en gran medida de la precisión y calidad de los datos ingresados. Las empresas deben asegurarse de que los datos sobre dimensiones, pesos y valores de los paquetes sean precisos y estén actualizados para obtener resultados óptimos.

\subsection{Capacidad de Escalabilidad}
El sistema debe ser capaz de manejar diferentes volúmenes de datos y escalar según las necesidades de la empresa. Un DSS que no puede escalar adecuadamente podría no ser útil en entornos de alta demanda o con grandes volúmenes de datos.

\subsection{Soporte Técnico y Mantenimiento}
La operación continua del DSS requiere un soporte técnico adecuado y mantenimiento regular. Las empresas deben considerar los recursos necesarios para mantener el sistema en funcionamiento y resolver cualquier problema técnico que pueda surgir.

\subsection{Consideraciones de Seguridad}
La seguridad de los datos es un aspecto crítico, especialmente cuando se integran sistemas que manejan información sensible sobre inventarios y logística. Las empresas deben implementar medidas de seguridad robustas para proteger los datos y garantizar la privacidad.

Superar estas barreras de entrada requiere una planificación cuidadosa, una evaluación exhaustiva de los costos y beneficios, y una gestión del cambio efectiva para asegurar que la implementación del DSS sea exitosa y aporte los beneficios esperados en términos de eficiencia y optimización de recursos.

\section{Conclusiones}
La implementación del prototipo del Sistema de Soporte de Decisiones (DSS) para la optimización del llenado de contenedores marítimos ha demostrado ser un paso importante hacia la mejora de la eficiencia en la utilización del espacio de carga. A través de este prototipo, se ha validado cómo un DSS basado en algoritmos genéticos y simulaciones de llenado manual puede aumentar significativamente el valor total de la carga y optimizar el espacio disponible.

El prototipo ha mostrado claramente cómo la tecnología puede asistir a los planificadores de carga en la toma de decisiones informadas. La capacidad del sistema para configurar parámetros del algoritmo y visualizar los resultados de la optimización ha permitido identificar áreas clave de mejora. Aunque el prototipo presenta limitaciones, como la entrada manual de datos y la visualización estática de los resultados, ha logrado cumplir su objetivo de servir como una prueba de concepto funcional.

El uso de herramientas como Shiny para Python y Matplotlib ha sido fundamental en la creación de una interfaz de usuario intuitiva y en la generación de gráficos efectivos. Estas herramientas han facilitado la presentación de los resultados de manera comprensible y accesible, demostrando el potencial del sistema para integrarse en procesos operativos reales.

A pesar de las limitaciones del prototipo, como la falta de importación de datos y la visualización estática, el proyecto ha demostrado que el DSS puede mejorar el llenado del contenedor, evidenciado por un aumento en el porcentaje de llenado y una optimización clara en la disposición de las cajas. Estas mejoras subrayan el valor potencial de un DSS completo y plenamente desarrollado para la logística y la gestión de carga en la industria marítima.

En resumen, el prototipo del DSS ha alcanzado sus objetivos demostrativos, mostrando el potencial de la tecnología para transformar procesos logísticos mediante la optimización avanzada y la toma de decisiones basada en datos. Los próximos pasos deberían centrarse en ampliar las capacidades del sistema, integrar más funcionalidades y realizar despliegues en entornos reales para validar su efectividad en operaciones diarias. Este proyecto sienta una base sólida para futuras mejoras y la eventual implementación de un sistema DSS completo y robusto.


%----------------------------------------------------------

\printbibliography[title={Referencias}]

%----------------------------------------------------------

\end{document}
\documentclass{beamer}

\title{Optimización del Problema de Carga Manual de Contenedores con Paquetes Heterogéneos}
\author{Jose G. Quilca y Javier Alcaraz}
\institute{Centro de Investigación Operativa, Universidad Miguel Hernández}
\date{\today}

\begin{document}

\frame{\titlepage}

\begin{frame}
    \frametitle{Introducción}
    \begin{itemize}
        \item Importancia del uso eficiente de contenedores en el comercio internacional.
        \item El problema de carga de contenedores es un desafío de optimización combinatoria NP-difícil.
        \item Enfoque en la carga manual por operadores, considerando restricciones realistas.
        \item Metaheurística basada en algoritmos genéticos.
    \end{itemize}
\end{frame}

\begin{frame}
    \frametitle{Objetivo}
    \begin{itemize}
        \item Mejorar el proceso de carga manual de contenedores.
        \item Desarrollar un algoritmo genético que optimice la configuración de carga.
        \item Validar la eficacia del método a través de experimentos computacionales.
    \end{itemize}
\end{frame}

\begin{frame}
    \frametitle{Revisión de Literatura}
    \begin{itemize}
        \item Historia y evolución del Problema de Carga de Contenedores (CLP).
        \item Clasificación de CLP: restricciones básicas y prácticas.
        \item Métodos de solución: heurísticos y metaheurísticos.
    \end{itemize}
\end{frame}

\begin{frame}
    \frametitle{Formulación del Problema}
    \begin{itemize}
        \item Contenedor con dimensiones y capacidad de carga fijas.
        \item Paquetes rectangulares de diferentes tamaños, pesos y valores.
        \item Restricciones de estabilidad, rotación y contigüidad.
    \end{itemize}
\end{frame}

\begin{frame}
    \frametitle{Procedimiento de Carga Manual}
    \begin{itemize}
        \item Secuencia de llenado: desde la parte trasera, inferior e izquierda del contenedor.
        \item Uso de algoritmos de metaheurística para simular el proceso de carga.
    \end{itemize}
\end{frame}

\begin{frame}
    \frametitle{Metaheurística Propuesta}
    \begin{itemize}
        \item Codificación de la solución con listas de secuencia de empaquetado, rotación y cantidad.
        \item Algoritmo genético: selección, cruce, mutación y elitismo.
    \end{itemize}
\end{frame}

\begin{frame}
    \frametitle{Estudio Experimental}
    \begin{itemize}
        \item Generación de datos de prueba: diferentes tipos de cajas y dimensiones.
        \item Implementación en Python y evaluación en múltiples instancias.
    \end{itemize}
\end{frame}

\begin{frame}
    \frametitle{Resultados y Análisis}
    \begin{itemize}
        \item Comparación de versiones mejoradas del algoritmo genético.
        \item Evaluación de calidad de solución y tiempo de ejecución.
    \end{itemize}
\end{frame}

\begin{frame}
    \frametitle{Conclusiones}
    \begin{itemize}
        \item Algoritmo genético eficiente para la carga de contenedores con paquetes heterogéneos.
        \item Mejoras significativas en calidad de solución y tiempo de convergencia.
        \item Futuras líneas de investigación: múltiples contenedores, más restricciones y optimización en tiempo real.
    \end{itemize}
\end{frame}

\begin{frame}
    \frametitle{Agradecimientos}
    \begin{itemize}
        \item Agradecimientos a Javier Alcaraz y al Instituto CIO por su apoyo.
        \item Agradecimientos a las familias y amigos por su constante apoyo.
    \end{itemize}
\end{frame}

\begin{frame}
    \frametitle{Preguntas y Discusión}
    \begin{center}
        \large{¿Preguntas?}
    \end{center}
\end{frame}

\end{document}

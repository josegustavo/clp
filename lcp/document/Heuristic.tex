\section{Heurística de carga manual}

\subsection{Descripción}

Para el problema planteado de Carga Manual de Paquetes en un Contenedor, se propone una heurística de carga con enfoque y restricciones propias en la que un operador humano realiza la carga de los paquetes en un contenedor de forma manual. La heurística propuesta se basa en la forma más simple y al mismo tiempo eficiente de cargar manualmente los paquetes en un contenedor, veremos en los siguientes apartados cómo se realiza la carga de los paquetes en el contenedor, así como también el algoritmo que se propone para la heurística de carga.

\subsection{Carga de paquetes en el contenedor}

Con el objetivo de facilitar la carga manual, se propone una serie de asunciones, restricciones y reglas que el operador humano debe seguir para realizar la carga de los paquetes en el contenedor de forma eficiente.

Se considera las siguientes asunciones sobre los paquetes:

\begin{itemize}
    \item Los paquetes son cajas de forma rectangular.
    \item Los paquetes pueden ser de diferentes tamaños, pesos y costos.
    \item Los paquetes con los mismos tamaños, pesos y costos son considerados como paquetes del mismo tipo.
    \item Los paquetes llegan agrupados según su tipo al contenedor y en determinado order.
    \item Los paquetes pueden ser apilados unos sobre otros sin importar su tipo, pero se debe garantizar la estabilidad de la carga.
    \item Cada tipo tiene un número de paquetes que deben ser cargados en el contenedor.
    \item Cada paquete debe mantener la misma orientación que los demás paquetes del mismo tipo.
    \item Los grupos de paquetes pueden llegar por bloques del mismo tipo o de forma secuencial, por ejemplo en cintas transportadoras.
\end{itemize}

Las asunciones relacionadas con el operador humano son las siguientes:

\begin{itemize}
    \item Uno o varios operadores humanos realizan la carga de los paquetes en un contenedor de forma manual.
    \item El operador humano tiene las indicaciones previas sobre la forma de cargar los paquetes en el contenedor, esto es el orden, la cantidad y orientación de cada tipo de paquete.
    \item Las indicaciones también incluyen los espacios que quedarán vacíos en el contenedor, los cuales pueden ser llenados con algún material de relleno con el objetivo de evitar que los paquetes se muevan durante el transporte.
\end{itemize}

Las indicaciones previas que son proporcionadas al operador humano son resultado de la solución del problema de optimización que se realiza. Estas asunciones y restricciones permiten que el procedimiento de carga manual libere al operador humano de la tarea de decidir la ubicación de los paquetes en el contenedor, permitiendo que se enfoque en seguir las indicaciones previas optimas para la carga de los paquetes.

\subsection{Algoritmo}

El algoritmo propuesto está basado en el método Deepest Bottom Left with Fill (DBLF) propuesto por Karabulut y otros \ref{karabulut2004hybrid}, cuyo use se ha extendido y varios autores han realizado propuestas para mejorarlo o adaptarlo a determinados contextos por ejemplo Wang y otros \ref{wang2010hybrid}, Kang y otros. \ref{kang2012hybrid}. El algoritmo propuesto en el presente trabajo está enfocado en cumplir las restricciones y adaptarse al contexto de una carga manual de paquetes en un contenedor.

La idea básica del algoritmo DBLF es que los paquetes comienzan a ser cargados en el contenedor de forma secuencial, se prioriza que el paquete se coloque en las siguientes posiciones: 

\begin{itemize}
    \item La posición más profunda del contenedor, que ayuda a llenar primero los espacios más alejados de la puerta y evitar obstaculizar el ingreso del operador humano al contenedor.
    \item La posición más baja del contenedor, se da naturalmente debido al efecto de la gravedad, un paquete no podría ser colocado en una posición alta si no se ha llenado primero las posiciones más bajas.
    \item La posición más a la izquierda del contenedor, aunque no es una restricción fuerte, podría usarse el criterio de priorizar la posición más a la derecha si se considera necesario, lo crucial es mantener la consistencia al elegir una de estas dos direcciones.
\end{itemize}

Al colocar cada paquete en el contenedor, el espacio restante se divide en tres nuevos sub-espacios, la parte superior, la parte lateral y la parte frontal, se elige la parte con mayor espacio restante para colocar el siguiente paquete respetando la prioridad DBLF indicada anteriormente en la figura \ref{fig:dblf} se muestra un ejemplo de cómo se divide el espacio restante en el contenedor al colocar un paquete.
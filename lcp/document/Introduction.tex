\section{Introducción} \label{sec: introducción}

En el comercio internacional, el transporte de mercancías se realiza principalmente a través de contenedores de carga. Los contenedores son cajas de acero de forma rectangular que se utilizan para transportar mercancías en barcos, trenes y camiones. Los contenedores son una forma eficiente y segura de transportar mercancías, ya que permiten que las mercancías se carguen y descarguen rápidamente y se almacenen de manera segura durante el viaje. Los contenedores vienen en diferentes tamaños y capacidades, y se utilizan para transportar una amplia variedad de mercancías, incluyendo productos manufacturados, materias primas, alimentos, etc. Para un buen aprovechamiento del espacio y la capacidad de carga de los contenedores, es importante que las mercancías se carguen de manera eficiente y se aproveche al máximo el espacio disponible.

En muchos casos, estas mercancías se encuentran en cajas o paquetes de diferentes tamaños, formas y pesos. Optimizar el llenado de dichos paquetes en los contenedores es un problema importante en la industria de la logística y el transporte ya que puede tener un impacto significativo en los costos y la eficiencia de la cadena de suministro. Por otro lado el mejor aprovechamiento del espacio y la capacidad de carga de los contenedores puede ayudar a reducir el número de viajes necesarios para transportar las mercancías, lo que puede reducir los costos de transporte y las emisiones de carbono asociadas \parencite{Parreo2008AMA}.

El problema de llenado de paquetes en contenedores es un problema de optimización combinatoria que ha sido ampliamente estudiado en la literatura. Así mismo, problemas similares se pueden observar en distintas industrias, como el llenado de paquetes en camiones, carga de pallets, carga en almacenes, entre otros, donde la colocación de cajas dentro de otras cajas más grandes es una tarea que se realiza con frecuencia. El llenado de contenedores consiste en colocar paquetes de diferentes tamaños y formas en un contenedor de manera que se utilice el espacio disponible de la mejor manera posible, cumpliéndose ciertas restricciones (peso, estabilidad, etc.) y optimizando uno o más objetivos, entre los que puede estar el minimizar el espacio no utilizado o maximizar el beneficio asociado a la carga transportada, por poner sólo algunos ejemplos.. Este problema ha sido clasificado como NP-duro \parencite{PISINGER2002382}, lo que significa que muchas veces para instancias grandes de paquetes no existe un algoritmo de tiempo polinomial que pueda resolverlo de manera exacta. Por ello, muchos autores han propuesto diferentes enfoques heurísticos y metaheurísticos para resolver este problema de manera aproximada.

Existen diferentes variantes del problema de la carga de contenedores con paquetes, dependiendo de las restricciones y objetivos específicos que se consideren. Algunas de las variantes más estudiadas incluyen el uso de paquetes homogéneos, paquetes heterogéneos, paquetes rotativos, paquetes frágiles, entre otros. En este trabajo, nos enfocaremos en restricciones derivadas de un caso de uso real que se da cuando la carga es realizada por uno o varios operarios, es decir una carga manual, cuyo principal objetivo es facilitar el proceso de la carga poniendo énfasis en las limitaciones que un operador humano pueda tener. Para esto se considera el uso de paquetes de baja heterogeneidad, que consiste en grupos de paquetes que comparten ciertas características, como el tamaño, el peso, el costo, es decir, paquetes que pueden ser cargados por una persona sin necesidad de maquinaria. También consideraremos restricciones de rotación, que indican que los paquetes pueden ser girados en ciertas direcciones para aprovechar mejor el espacio disponible y restricciones de contigüidad, que indican que los paquetes del mismo grupo deben ser cargados de manera contigua.

En este trabajo, se propone una metaheurística basada en el algoritmo genético para resolver el problema de la carga manual de contenedores con paquetes heterogéneos. El algoritmo genético es una técnica de optimización que se basa en la evolución biológica y que ha sido ampliamente utilizada para resolver problemas de optimización combinatoria. El algoritmo genético es un enfoque de búsqueda poblacional que mantiene una población de soluciones candidatas y utiliza operadores genéticos como la selección, el cruce y la mutación para generar nuevas soluciones a partir de las soluciones existentes. El algoritmo genético es un enfoque flexible y versátil que ha demostrado ser efectivo para resolver una amplia variedad de problemas de optimización combinatoria.

Para evaluar las soluciones que el algoritmo genético va generando, se ha implementado un algoritmo que simula el llenado manual que realizaría un operario, teniendo en cuenta todos los condicionantes que supone este proceso. Por ejemplo, el operario no puede alcanzar ciertas zonas del contenedor si no hay espacio suficiente para pasar o si otros paquetes que han sido colocados anteriormente le obstruyen el paso.

Este trabajo se enfoca en resolver el problema de llenado de contenedores considerando un único contenedor y paquetes débilmente heterogéneos, esta clasificación también es conocida como Three-dimensional Single Large Object Placement Problem (SLOPP) cuya clasificación fue propuesta por \textcite{WASCHER20071109}, sin embargo muchos de los trabajos en la literatura que resuelven este problema no consideran todas restricciones prácticas que se presentan en la industria. \parencite{SAFAK2023106199} también mencionan que encontrar una disposición óptima de los paquetes en los contenedores mientras se satisfacen una variedad de restricciones del mundo real es una tarea no trivial.

Con esta falta de enfoques en la literatura con un conjunto realista de restricciones prácticas, nuestro objetivo es proponer un método metaheurístico de buen rendimiento que sea capaz de resolver problemas con un número razonable de número de paquetes y garantizar resultados consistentes de buena calidad. Consideramos las restricciones prácticas que tienen como origen del hecho de ser un llenado manual de los paquetes en el contenedor. El método de solución propuesto consiste en un enfoque de llenado priorizando el espacio más al fondo, más debajo y más a la izquierda, compuesto por un simulación de llenado por ordenador y unas propuestas de mejoras en los procesos de llenado. El método está estructurado de tal manera quede claro y facilite el procedimiento de carga para el operario.

Las contribuciones de este trabajo son las siguientes:

\begin{itemize}
    \item Se propone un método de solución basado en una metaheurística de buen rendimiento para resolver el problema de llenado de contenedores con un conjunto realista de restricciones prácticas derivadas de un caso de uso real de carga manual.
    \item Se propone una mejora en el proceso de llenado que ayuda a la metaheurística a encontrar soluciones de mejor calidad en menos tiempo.
    \item Se presenta una serie de experimentos computacionales que muestran la eficacia del método propuesto en diferentes instancias del problema.
\end{itemize}

El resto de este trabajo está organizado de la siguiente manera. En la sección 2, se presenta una revisión de la literatura relacionada con el problema de la carga de contenedores y sus variantes. En la sección 3, se presenta la definición del problema en particular. En la sección 4, se describe la metaheurística propuesta para resolver el problema. En la sección 5, se presenta un estudio computacional para evaluar el desempeño con diferentes configuraciones. Finalmente, en la sección 6, se presentan las conclusiones y las direcciones futuras de investigación.
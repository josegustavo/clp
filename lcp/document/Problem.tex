\section{Formulación del problema}

Siendo un contenedor de ancho $W$, largo $L$, alto $H$, y una capacidad de carga $P$,
se tiene definido unos tipos de paquetes $t \in T = \{0, 1, 2, \ldots, n\}$, donde cada tipo $t$ posee ciertas dimensiones de ancho $w_t$, largo $l_t$, alto $h_t$, también posee un peso $p_t$ y un costo $c_t$, además se conoce la cantidad máxima de paquetes por cada tipo $q_t$ que un contenedor puede tener.
En este problema, consideramos que $W$, $L$, $H$ y $P$ , $w_t$, $l_t$, $h_t$, $p_t$, $q_t$ son enteros positivos.

Se tienen restricciones de rotación debido al enfoque de carga manual, en el cual se establece que $\forall r \in r_t, r \in \{0, 1\}, t \in T$ donde $0$ representa que el tipo no se encuentra girado y $1$ que el tipo está girado 90 grados en el eje $x$, lo que implica que los anchos y largos pueden intercambiarse, mientras que la altura no puede ser modificada.

Por otro lado para facilitar la carga manual, se debe disponer de un orden donde cada paquete del mismo tipo debe ser cargado de manera contigua.

El problema consiste en determinar la cantidad de paquetes por cada tipo a cargar $\tilde{q}_t$ (la cual no puede superar al máximo establecido por tipo, $0 \leq \tilde{q}_t \leq q_t$) y el orden de carga de cada tipo $o_t$ con determinada rotación $r_t$, de tal modo que se pueda obtener la disposición óptima de los paquetes en el contenedor, asegurando el cumplimiento de las restricciones relacionadas al espacio disponible y el peso. Además, se busca maximizar la utilización del espacio del contenedor y al mismo tiempo el costo de la carga.
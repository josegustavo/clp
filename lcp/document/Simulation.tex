\section{Simulación de llenado manual}

Para resolver el problema de la carga manual de paquetes en un contenedor, esta sección presenta un procedimiento específico que incluye restricciones orientadas a facilitar la tarea del operador humano. Este método está diseñado para ser la forma más sencilla y eficiente de organizar los paquetes dentro del contenedor, con el fin de estandarizar el proceso de carga y minimizar la carga cognitiva del operador reduciendo la necesidad de tomar decisiones sobre la disposición de los paquetes. Este proceso actúa como un puente entre la solución de optimización y la intervención manual del operador, proporcionando instrucciones claras y precisas para una carga eficiente.

En las secciones siguientes, detallaremos cómo se efectúa la carga de los paquetes, las suposiciones y restricciones que se consideran, luego presentaremos el algoritmo de llenado manual. Este algoritmo se basa en el método Deepest Bottom Left with Fill (DBLF) propuesto por Karabulut y otros \textcite{karabulut2004hybrid}, el cual ha sido adaptado para cumplir con las restricciones de un llenado manual de contenedores.

\subsection{Procedimiento de carga manual}

Para estandarizar la carga manual y facilitar la simulación por computadora, así como simplificar el trabajo del operador, se han establecido una serie de supuestos, restricciones y reglas que deben seguirse para optimizar la carga de paquetes en el contenedor.

En relación con la definición del problema, se resume las siguientes suposiciones acerca de los paquetes:

\begin{itemize}
    \item Los paquetes son cajas de forma rectangular.
    \item Los paquetes pueden variar en tamaño, peso y valor.
    \item Los paquetes presentan tamaños y pesos razonables para ser cargados manualmente.
    \item Los paquetes que comparten el mismo tamaño, peso y valor estrictamente se consideran del mismo tipo.
    \item Dos paquetes pueden tener el mismo tamaño y peso pero distinto valor, lo que los convierte en tipos diferentes.
    \item El valor de un paquete no depende de su tamaño o peso, es decir que un paquete sea más grande y pesado que otro no implica que sea de mayor valor y viceversa.
    \item Los paquetes llegan al contenedor agrupados por tipo y en un orden específico.
    \item Los paquetes pueden apilarse unos sobre otros, independientemente de su tipo, pero se debe asegurar la estabilidad de la carga.
    \item Cada tipo de paquete tiene una cantidad fija deseada que debe ser cargada en el contenedor.
    \item Todos los paquetes de un mismo tipo deben mantener la misma orientación.
    \item Los grupos de paquetes llegan en bloques del mismo tipo o de forma secuencial, por ejemplo, a través de cintas transportadoras.
\end{itemize}

Las suposiciones relacionadas con el operador humano son las siguientes:

\begin{itemize}
    \item Uno o varios operadores humanos realizan la carga de los paquetes en el contenedor de forma manual.
    \item El operador humano recibe indicaciones previas sobre cómo cargar los paquetes en el contenedor, incluyendo el orden, la cantidad y la orientación de cada tipo de paquete.
    \item Las indicaciones también podrían especificar los espacios que deberán quedar vacíos en el contenedor, los cuales pueden ser llenados con material de relleno para evitar que los paquetes se muevan durante el transporte.
    \item Las indicaciones previas proporcionadas al operador humano son el resultado de la solución del problema de optimización de la carga.
    \item El objetivo del operador humano es seguir las indicaciones previas de manera eficiente y precisa, sin necesidad de tomar decisiones adicionales sobre la disposición de los paquetes.
\end{itemize}

El procedimiento de carga manual se basa en la combinación de las suposiciones, restricciones y reglas mencionadas anteriormente, con el objetivo de lograr una carga eficiente y organizada de los paquetes en el contenedor. Este procedimiento se implementa siguiendo una metodología específica que guía al operador humano en la colocación de los paquetes, asegurando que se cumplan todas las condiciones establecidas.

Para resolver esta primera parte problema, se propone un algoritmo de llenado manual que guía al operador humano en la colocación exacta de los paquetes en el contenedor, siguiendo un orden específico y respetando las restricciones de rotación y orientación de los paquetes.



\subsection{Conclusiones}

En este capítulo se ha detallado la problemática, también se ha descrito una propuesta de solución, se ha especificado las restricciones, reglas y suposiciones para el problema, formulando una representación matemática, así como una propuesta de algoritmo de llenado manual basado en el método Deepest Bottom Left with Fill (DBLF) adaptado a las restricciones de un llenado manual de contenedores. Se han propuesto mejoras al algoritmo DBLF, como la unión de subespacios, la eliminación de subespacios inaccesibles y la eliminación de subespacios profundos, para adaptarlo a las restricciones de un llenado manual de contenedores.

En el siguiente capítulo se presentará un algoritmo de optimización genético para resolver el problema de optimización de la carga manual de paquetes en un contenedor, considerando las restricciones de espacio del contenedor, así como las restricciones de rotación y orientación de los paquetes.
